\section{Introduction}
The motivation for creating a package for the modelling of smooth transition
ARMA models was the observation that many phenomena appear to go
through different states which may be explained by some underlying and
observable factors.\footnote{As opposed to unobservable factors which leads to
a different class of models, such as the Markov switching models.}
In financial markets, under different states of the business cycle, financial
instruments have been observed to exhibit different characteristics with
recessions (or the lead up to such) usually marked by increased volatility and lower 
or negative mean. Being able to model the evolution of the process driving such 
changes and hence the switch from one state to another must surely make for a 
better understanding of the underlying dynamics and perhaps lead to a better
forecast model.

The precursor to smooth transition models appears to have been
\cite{Carmichael1928} who posited the use of the arctangent transformation
and even considered the possibility of a double transition well before the
plethora of papers which came more than 30 years later. While \cite{Quandt1958}
originally discussed a switching regression model, the pioneering contributions
to the literature on more general threshold autoregression (TAR) have
been \cite{Tong1980} and \cite{Tong1981}, with a more general class of nonlinear
AR models introduced in a series of papers by \cite{Billings1983}, \cite{Billings1986},
and \cite{Zhu1993}. Many extensions to the basic TAR model have been considered
in the literature , including the threshold moving average model in
\cite{Gooijer1998}, threshold ARMA processes in \cite{Brockwell1992}, smooth
transition based on the Gaussian CDF in \cite{ChanTong1986}, the more widely
adopted logistic (LSTAR) and exponential (ESTAR) models discussed in
\cite{Terasvirta1994}, double transitions in \ldots , nested transition
models in \cite{Astatkie1997}, multiple states in \cite{Dijk1999}, and the
inclusion of GARCH dynamics in \citeauthor{Chan2002}(\citeyear{Chan2002}, 
\citeyear{Chan2003}). Thresholds in the variance dynamics have been considered
by \cite{Zakoian1994} for GARCH models, and \cite{So2002} for stochastic
volatility models. A more general class of hierarchical mixture time series models 
was proposed by \cite{Huerta2003}, while \cite{Kalliovirta2012} proposed a 
Gaussian mixture AR type model. A type of mixture model is also available in
the \textbf{twinkle} package and discussed in the next section. For a general 
review of 'recent' extensions and the state of research in this area
see \cite{Dijk2002}.

A common theme among the vast majority of econometric research in the area of
smooth transition AR models has been the use of the self-exciting model,
where the lagged value of the dependent variable (or some simple transformation 
of the same) is used. The focus on this model has also restricted the
representation used in most papers to the one proposed originally by
\cite{Terasvirta1994} which models the state transition rate ($\gamma$), or
scaling factor, seperately from the dynamics. The \textbf{twinkle} package
departs from this represenation and reparameterizes the state
dynamics to include a possible linear combination of multivariate
variables and the use of autoregressive first order dynamics. The m-states model
is also reparameterized to more closely resemble the representation of the
multinomial logistic regression model in the way the probabilities are summed
and weighted across states. Further extensions include a 2-state AR mixture
model partially bridging the gap with the finite mixture models, the inclusion of 
GARCH dynamics, MA dynamics either inside or outside the states, and a large
number of conditional distributions which follow from the rugarch package. Methods 
for model specification, estimation, filtering, forecasting and simulation are
provided with similar interface and access methods as in related packages by the author. 

It is important for the interested user to be aware from the start that such
models are difficult to estimate, may contain local minima and may be generally
hard to solve with confidence. While the package has made efforts to provide for a 
number of solvers and strategies to estimate these models, confident estimation
may prove challenging depending on properties of the dataset used and choice of 
model options. The model is naturally greedy in requiring a substantial amount of data to
confidently identify the optimal classification of states given the conditional
mean equation. From experience, it is this author's opinion that these types of
models may not be as forgiving as linear models when it comes to forecasting, 
depending on whether the actual forecast state contains the type of
nonlinearities under which the model was estimated. Thus, unlike linear
models, the misclassification of the forecast state may be more costly.

This paper is organized as follows: Section \ref{sec:1} discusses the
representation of the model in the \textbf{twinkle} package and how it can be
specified. Section \ref{sec:2} discusses forecasting with a special emphasis on
the different methods implemented for n-period ahead forecasts, followed by 
Section \ref{sec:3} on simulation. Finally, Section \ref{sec:4} presents a 
number of examples using real and simulated data.

While every possible effort has been made to test the model and its methods
under different scenarios and squash any bugs, the package is still quite new
and further testing is required. As in related packages, a test folder with
examples/tests is available in the \emph{inst} folder of the source
distribution. The package is available to download from the Bitbucket
repository (\emph{alexiosg}) which also contains the development versions
of other packages by the author.

General questions on the package should be posted to the R-SIG-FINANCE mailing
list, while bugs (with reproducible code and preferably a patch) and 
suggestions can be reported directly to me.

Finally I would like to acknowledge the valuable help of Eduardo Rossi who
collaborated on the new representation and research publication in this area.

\section{Smooth Transition ARMA Models Revisited}\label{sec:1}
\subsection{Dynamics and Extensions}
Consider the standard representation of a 2-state STAR model (adapted from
\cite{Dijk1999})
\begin{equation}\label{eq:star_original}
{y_t} = {{\phi '}_1}y_t^{\left( p \right)}\left( {F\left( {{z_t};\gamma
,\alpha ,c} \right)} \right) + {{\phi '}_2}y_t^{\left( p \right)}\left( {1 - F\left( {{z_t};\gamma ,\alpha ,c} \right)} \right) + {\varepsilon _t}
\end{equation}
where $y_t^{\left( p \right)} = {\left( {1,\tilde y_t^{\left( p
\right)}}\right)^\prime }$, $\tilde y_t^{\left( p \right)} = {\left(
{{y_{t-1}},\dots,{y_{t - p}}} \right)^\prime }$,${\phi _i} = {\left( {{\phi
_{i0}},{\phi _{i1}},\dots,{\phi _{ip}}} \right)^\prime }$, $i = 1,2$ (states)
and $\varepsilon_t$ is a white noise zero mean error process with standard deviation
$\sigma$. The state transition function $F\left( {{z_t};\gamma ,\alpha ,c} \right)$ 
is a continuous function bounded on the unit interval and usually taken to be
the logistic CDF\footnote{At present only the Logistic STAR model is
entertained and it is not likely that the exponential STAR model will be
considered at all.} such that:
\begin{equation}\label{eq:logistic_cdf}
F\left( {{z_t};\gamma ,\alpha ,c} \right) = {\left( {1 + \exp \left\{ { -
\gamma \left( {\alpha '{z_t} - c} \right)} \right\}} \right)^{ - 1}},\gamma  > 0
\end{equation}
where ${z_t} = {\left( {{z_{1t}},\dots,{z_{jt}}} \right)^\prime },j = 1,\dots,k$
is a vector of $k$ observed variables which are assumed to explain the state
transition. These can be a set of explanatory variables or the lagged values of $y_t$ in which case
the model is called 'self-exciting'. It is also possible that the variable is
time in which case the model can be used to identify breaks in the mean as in 
\cite{Lin1994}, or a combination of time and other variables giving rise to the
time varying STAR (TVSTAR) model discussed in \cite{Lundbergh2003}. As correctly 
noted by \cite{Dijk1999}, the vector of parameters $\alpha$ needs to be
nornalized in some way in order to achieve identification (i.e by setting
$\alpha_1$=1). The parameter $\gamma$ is then a type of scaling factor which
determines the smoothness (or speed) of the transition, with values at the limits, 
$\left[ {0,\infty } \right]$, representing linear and TAR type
transitions respectively. By far the most popular test of STAR nonlinearity is described 
in \cite{Luukkonen1988} using a Taylor series expansion around
equation \ref{eq:logistic_cdf}, effectively testing whether $\gamma=0$ (via
a auxilliary regression), which would in turn imply that the $\alpha$ vector is
also zero and hence a rejection of STAR type non-linearity.
However, there is really little reason for estimating $\gamma$ seperately in the
STAR model since we can allow it to be subsumed by the vector of state 
parameters $\left(\alpha, c \right)$. In doing so we also gain the additional
advantage of extending the type of dynamics to include autoregression as
follows:
\begin{equation}\label{eq:star_new}
\begin{gathered}
  F\left( {{z_t};\alpha ,c,\beta} \right) = {\left( {1 + \exp \left\{ -{{\pi
  _t}} \right\}} \right)^{ - 1}} \hfill \\
  {\pi _t} = \hat c + \hat \alpha '{z_t} + \beta '\pi _t^{\left( q \right)} \hfill \\
  \pi _t^{\left( q \right)} = {\left( {{\pi _{t - 1}},\dots,{\pi _{t - q}}}
  \right)^\prime } \hfill \\
\end{gathered}
\end{equation}
where the unconstrained state dynamics $\pi_t$ can be initialized by setting
${\pi _0} =\frac{{\hat c + \hat \alpha 'E\left[ z \right]}}{{1 - \beta
'{\mathbf{1}}}}$ with a stationarity constraint of $\left| {\beta
'{\mathbf{1}}} \right| < 1$.
It should be  clear from this representation that  $\hat c = \gamma c,\hat
\alpha ' = \gamma {\left( {1,{\alpha _2},\dots,{\alpha _j}} \right)^\prime },j =
1,\dots,k$ recovers the  original representation in equation
\ref{eq:star_original}. The use of  autoregressive dynamics in the state
equation follows related work in the  area of dynamic binary response models of 
\cite{Kauppi2008} and \cite{Nyberg2010}.
Generally, estimation becomes quite difficult for more than one autoregressive
parameter in the state dynamics which is why at present only a lag-1
autoregressive state dynamics model is allowed in the package.

The conditional mean dynamics are not limited to AR terms but may include
external regressors (ARX) and moving average (MA) terms in the states giving
rise to a full STARMAX model specification:
\begin{equation}\label{eq:starmax}
\begin{split}
{y_t} &= \left( {{{\phi '}_1}y_t^{\left( p \right)} + {{\xi '}_1}{x_t} + {{\psi
'}_1}e_t^{\left( q \right)}} \right)\left( {F\left( {{z_t};\alpha ,c,\beta }
\right)} \right) \\
& + \left( {{{\phi '}_2}y_t^{\left( p \right)} + {{\xi
'}_2}{x_t} + {{\psi '}_2}e_t^{\left( q \right)}} \right)\left( {1 - F\left( {{z_t};\alpha ,c,\beta } \right)} \right) + {\varepsilon _t}
\end{split}
\end{equation}
where $\varepsilon _t^{\left( q \right)} = {\left( {{\varepsilon _{t -
1}},\dots,{\varepsilon _{t - q}}} \right)^\prime },{{\psi '}_i} = {\left( {{\psi
_{i1}},\dots,{\psi _{iq}}} \right)^\prime }$ represent the $q$ moving average
terms and parameters per state $i$ ($i=1,2$), and ${x_t} = {\left(
{{x_1},\dots,{x_l}} \right)^\prime },{{\xi '}_1} = {\left( {{\xi
_{i1}},\dots,{\xi _{il}}} \right)^\prime }$ the $l$ external regressors and
their parameters per state $i$. It is also possible that the MA term enters
outside of the states instead of inside giving rise to a STARX with Linear MA
terms (STARXLMA):
\begin{equation}
{y_t} = \left( {{{\phi '}_1}y_t^{\left( p \right)} + {{\xi '}_1}{x_t}}
\right)\left( {F\left( {{z_t};\alpha ,c,\beta } \right)} \right) + \left(
{{{\phi '}_2}y_t^{\left( p \right)} + {{\xi '}_2}{x_t}} \right)\left( {1 -
F\left( {{z_t};\alpha ,c,\beta } \right)} \right) + \psi 'e_t^{\left( q \right)}
+{\varepsilon _t}
\end{equation}
The STARMAX model therefore encompasses a very wide range of sub-models based on
the type of restrictions placed in the conditional mean and state dynamics, and
choice of switching variables in the latter.\\
A natural question which arises from the representation is whether it is
reasonable to assume that the conditional variance is the same in both states.
Re-write the 2-state STARX equation as follows:
\begin{equation}
{\varepsilon _t} = {y_t} - {p_t}\left( {{\mu _{1t}}} \right) - \left( {1 -
{p_t}} \right)\left( {{\mu _{2t}}} \right)
\end{equation}
where $\mu_{i1}$ and $\mu_{2,t}$ represent the conditional mean dynamics per
state at time $t$ and $p_t$ the conditional probability. Add and subtract
${p_t}{y_t}$, and re-arrange:
\begin{equation}\label{eq:starmix}
\begin{gathered}
  {\varepsilon _t} =  {\color{blue}+ {p_t}{y_t}} - {p_t}\left( {{\mu _{1t}}}
  \right) + {y_t} {\color{blue} - {p_t}{y_t}} - \left( {1 - {p_t}} \right)\left(
  {{\mu _{2t}}} \right) \hfill \\
  {\varepsilon _t} = {p_t}{y_t} - {p_t}\left( {{\mu _{1t}}} \right) + {y_t}\left( {1 - {p_t}} \right) - \left( {1 - {p_t}} \right)\left( {{\mu _{2t}}} \right) \hfill \\
  {\varepsilon _t} = {p_t}\left( {{y_t} - {\mu _{1t}}} \right) + \left( {1 - {p_t}} \right)\left( {{y_t} - {\mu _{2t}}} \right) \hfill \\
  {\varepsilon _t} = {p_t}\left( {{\varepsilon _{1,t}}} \right) + \left( {1 - {p_t}} \right)\left( {{\varepsilon _{2,t}}} \right) \hfill \\
  {\varepsilon _{1,t}} \sim N\left( {0,\sigma _1^2} \right) \hfill, {\varepsilon
  _{2,t}} \sim N\left( {0,\sigma _2^2} \right) \hfill \\
  {\varepsilon _t} \sim N\left( {0,{p_t}\sigma _1^2 + \left( {1 - {p_t}}
  \right)\sigma _2^2} \right) \hfill \\
\end{gathered}
\end{equation}
Thus, the model can naturally be re-formulated as a mixture of
Normals\footnote{Or any other location-scale invariant distribution} with
mixing probabilities derived from the state dynamics.
This provides for a more parsimonious and clear extension than using GARCH dynamics on the mixed state
residuals. Alternatively, it can be thought of as the time-invariant version of
a STARX-STGARCH model with common transition dynamics. This also provides a
partial bridge between finite mixture and time-series autoregressive models.

\subsection{Multiple States}
\cite{Dijk1999} considered an extension of the 2-state STAR model in equation
\ref{eq:star_original} to a 4-state models as follows:
\begin{equation}\label{eq:mrstar_original}
\begin{aligned}
{y_t} &=& \left[ {{{\phi '}_1}y_t^{\left( p \right)}\left( {1 - F\left(
{{z_t};{\gamma _1},\alpha ,c} \right)} \right) + {{\phi '}_2}y_t^{\left( p
\right)}\left( {1 - F\left( {{z_t};{\gamma _1},\alpha ,c} \right)} \right)}
\right]\left( {1 - F\left( {{z_t};{\gamma _2},b,d} \right)} \right)\\
&+& \left[ {{{\phi '}_3}y_t^{\left( p \right)}\left( {1 - F\left(
{{z_t};{\gamma _1},\alpha ,c} \right)} \right) + {{\phi '}_4}y_t^{\left( p \right)}\left( {1 -
F\left( {{z_t};{\gamma _1},\alpha ,c} \right)} \right)} \right]F\left(
{{z_t};{\gamma _2},b,d} \right) + {\varepsilon _t}\\
\end{aligned}
\end{equation}
Alternatively, the implementation followed in this package takes a
page out of multinomial regression and models multiple states using the
following representation:
\begin{equation}\label{eq:mrstar_new}
{y_t} = \sum\limits_{i = 1}^s {\left[ {\left( {{{\phi '}_i}y_t^{\left( p \right)} + {{\xi '}_i}{x_t} + {{\psi '}_i}e_t^{\left( q \right)}} \right){F_i}\left( {{z_t};{\alpha _i},{c_i}} \right)} \right]}  + {\varepsilon _t}
\end{equation}
with
\begin{equation}
\begin{gathered}
  {F_i}\left( {{z_t};{\alpha _i},{c_i}} \right) = \frac{{{e^{{\pi _{i.t}}}}}}
{{1 + \sum\limits_{i = 1}^{s - 1} {{e^{{\pi _{i,t}}}}} }} \hfill \\
  {F_s}\left( {{z_t};{\alpha _i},{c_i}} \right) = \frac{1}
{{1 + \sum\limits_{i = 1}^{s - 1} {{e^{{\pi _{i,t}}}}} }} \hfill \\ 
\end{gathered}
\end{equation}
where the $s$ states are weighted to sum to unity. This appears, at least to
this author, to be a more natural representation for a multi-state setup. In the
\textbf{twinkle} package, upto 4 states may be modelled\footnote{For the case of
only a single state, it is also possible to pass a set of 'time' weights.}

\subsection{Estimation}
Estimation of the STARMAX models is done by maximizing the likelihood without
imposing any particular inequality restrictions on the state dynamic intercepts
or any parameter bound restrictions (except for positivity bounds on the
variance).\footnote{Constraints on the autoregressive and moving average
parameters may also be placed to impose some type of stationarity constraint
per state.} Since unconstrained optimizers appear to do quite well for hard
nonlinear/non-smooth problems, the main solver in the twinkle package is the
BFGS solver from the optim function. It is possible to include parameter bounds
in which case a logistic transformation is used with the unconstrained solvers.
Additional solvers included are 'nlminb' (bound constrained), 'solnp'
(nonlinear SQP solver with nonlinear constraints), 'cmaes' (bound constrained
global solver) and 'deoptim' (bound constrained global solver).
However, it is suggested that either a multi-start strategy is followed
(by choosing 'msoptim') or an iterative search strategy ('strategy') which
cycles between fixing the state parameters to estimate the conditional mean 
parameters (linear), fixing the conditional mean parameters to estimate the
state parameters (nonlinear) and a random start estimation. As with general
nonlinear optimization problems, scaling of the variables prior to estimation
may help, an ica or pca transformation if they are highly correlated, or hinge
basis transformation of the dataset via the earth package is another interesting
option in the case of relevant feature extraction.
\section{Forecasting}\label{sec:2}
Consider a general nonlinear first order autoregressive model:
\begin{equation}\label{eq:nonlinar1}
{y_t} = F\left( {{y_{t - 1}};\theta } \right) + {\varepsilon _t}
\end{equation}
where $ F\left( {{y_{t - 1}};\theta } \right)$ is some nonlinear function 
mapping $y_{t-1}$ to $y_t$ given the parameter set $\theta$. The optimal h-step
ahead  point forecast, using a least squares criterion, of $y_{t+h}$ at time $t$
is given by:
\begin{equation}
{{\hat y}_{t + h\left| t \right.}} = E\left[ {{y_{t + h}}\left| {{\Im _t}} \right.} \right]
\end{equation}
where $\Im_t$ is the information set upto time $t$. Given  that $E\left[
{{\varepsilon_{t + 1}}\left| {{\Im _t}} \right.} \right] = 0$, then the
1-step-ahead optimal forecast is:
\begin{equation}
{{\hat y}_{t + 1\left| t \right.}} = E\left[ {{y_{t + 1}}\left| {{\Im _t}} \right.} \right] = F\left( {{y_t};\theta } \right)
\end{equation}
which is the same as when $F(.)$ is linear. However, for horizons greater than
1, this is not the case since $E\left[ {F\left( . \right)} \right] \ne F\left(
{E\left[ . \right]} \right)$,  which means that simple recursive relationship 
found in the linear case do not exist in the nonlinear case. Instead, consider
the  h-step-ahead point forecast using the following closed form 
representation:\footnote{This is based on the Chapman-Kolmogorov relation:
\begin{equation}
g\left( {{y_{t + h}}\left| {{\Im _t}} \right.} \right) = \int_{ - \infty }^\infty  {g\left( {{y_{t + h}}\left| {{y_{t + h - 1}}} \right.} \right)g\left( {{y_{t + h - 1}}\left| {{\Im _t}} \right.} \right)d{y_{t + h - 1}}}
\end{equation}
which leads to Equation \ref{eq:hstepforecast} after taking conditional expectations from both sides.
}
\begin{equation}\label{eq:hstepforecast}
E\left[ {{y_{t + h}}\left| {{\Im _t}} \right.} \right] = \int_{ - \infty }^\infty  {E\left[ {{y_{t + h}}\left| {{y_{t + h - 1}}} \right.} \right]g\left( {{y_{t + h - 1}}\left| {{\Im _t}} \right.} \right)d{y_{t + h - 1}}}
\end{equation}
where $g\left( {{y_{t + h}}\left| {{\Im _t}} \right.} \right) = f\left( {{y_{t + h}} - F\left( {{y_{t + h - 1}};\theta } \right)} \right)$, is the distribution of the shock $\varepsilon_{t+h}$ with mean $F\left( {y_{t + h - 1}}\right)$, though the distribution $\varepsilon_t$ is never known with certainty.
A number of approaches have been used in the literature to estimate this
integral. It is simple to see that the conditional distribution of ${g\left(
{{y_{t + h - 1}}\left| {{\Im _t}} \right.} \right)}$ can be obtained recursively
starting at h=2 and noting that $g\left( {{y_{t + 1}}\left| {{\Im _t}} \right.}
\right) = f\left( {{y_{t + 1}} - F\left( {{y_t};\theta } \right)} \right)$. To
obtain the forecasts, numerical integration can be used (applied recursively) or
monte carlo methods. In the former case, the form of the conditional
distribution $f\left( {.\left| {{\Im _t}} \right.} \right)$ can be replaced by a
kernel estimator, whereas in the latter case one has an option of using an
empirical bootstrap, simulating from the conditional distribution $f\left(
{.\left| {{\Im _t}} \right.} \right)$ or a kernel estimator.  For instance,  
the 2-step ahead monte carlo forecast is given by:
\begin{equation}
{{\hat y}_{t + 2\left| t \right.}} = \frac{1}{T}\sum\limits_{i = 1}^T {F\left( {{{\hat y}_{t + 1\left| t \right.}} + {\varepsilon _i};\theta } \right)}
\end{equation}
However, in the case when a GARCH model is used for the modelling of the
conditional variance, then the monte carlo forecast needs to be adjusted as follows:
\begin{equation}
{{\hat y}_{t + 2\left| t \right.}} = \frac{1}{T}\sum\limits_{i = 1}^T {F\left( {{{\hat y}_{t + 1\left| t \right.}} + {z_i}{{\hat \sigma }_{t + 2\left| t \right.}};\theta } \right)}
\end{equation}
where $z_i$ represent draws from either the parametric standardized distribution
of the model  or the standardized in-sample innovations (or draws from a kernel
estimated density of the standardized in-sample innovations), which are then
multiplied  by the forecast GARCH volatility ${{{\hat \sigma }_{t + 2\left| t
\right.}}}$ to obtain the forecast residuals $\varepsilon_i$.
One benefit of using a monte-carlo or bootstrap approach is that they
immediately give rise to the density of each point forecast thus allowing for
the creation of interval forecasts.

\section{Simulation}\label{sec:3}
The key issue considered in the implementation of a simula

\section{Generalized Impulse Response}\label{sec:4}

\pagebreak
\section{Software Implementation}\label{sec:5}
\subsection{Specification}
The entry point to defining and estimating a STARMAX model in the twinkle
package is the starspec function:
\begin{lstlisting}
>starspec
function(
mean.model = list(states=2, include.intercept=c(1,1), arOrder=c(1,1), 
	maOrder=c(0, 0), matype="linear", statevar=c("y","s"), s=NULL, 
	statear=FALSE, ylags=1, xreg=NULL, yfun=NULL, transform="log"), 
variance.model=list(dynamic=FALSE, model="sGARCH", garchOrder=c(1,1), 
	submodel=NULL, vreg=NULL, variance.targeting=FALSE), 
distribution.model="norm", start.pars=list(), fixed.pars=list(), 
fixed.prob=NULL, ...)
}
\end{lstlisting}
The \textbf{mean.model} defines the equation for the conditional mean dynamics
including the state dynamics. Upto 4 \textbf{states} are allowed, with the
1-state option having a special implementation in that the \textbf{fixed.probs}
list is an xts matrix (alined to the dataset which will be passed to the
estimation routine) of weights. By default this is set to a vector of ones in
this case but may be any other 'time-weighting' scheme the user wishes. The
options for \textbf{intercept}, \textbf{arOrder} and \textbf{maOrder} should be
integer vectors of length equal to the number of \textbf{states}. 

The \textbf{matype} denotes whether the moving average terms enter inside the
states ('states') or outside ('linear'). 

The \textbf{statevar} indicates whether the model will switch based on its own
value ('y') or an exogenouse set of regressors ('s'), in which case an xts 
matrix (aligned to the index of the dataset and appropriately lagged) is passed 
to \textbf{s}. In the case that \textbf{statevar} is 'y', then \textbf{ylags} 
is an integer vector of the unique lags to use as a linear combination. 

The \textbf{yfun}  option allows the user to pass a function to transform the 
value of y\footnote{The function must return the same length as the value  it
receives without any NAs or NaNs.} prior to being used in the state dynamics 
equation. While it may appear at first that the same can be achieved by passing 
a pre-transformed value and using 's' as the \textbf{statevar}, consider that
simulation and n-ahead forecasts (which depend on simulation methods) on 
transformed values of 'y' can then be used directly, where it would have been
impossible to do so otherwise (because of the path dependency). 

The \textbf{xreg} is an optional xts matrix of external regressors which needs
to be aligned to the index of the dataset (and appropriately lagged). The
\textbf{statear} indicates whether to include lag-1 autoregression in the state
dynamics as discussed previously in equation \ref{eq:star_new}.

Tthe \textbf{transform} is currently fixed to use only the logistic
transformation, and there are no plans to extend to the exponential at present.

The variance model can be \textbf{dynamic}, in which case a choice of
'mixture','sGARCH','gjrGARCH' and 'eGARCH' are implemented, else by default a
static variance model is used. For the GARCH flavors, the rest of the options
follow from the rugarch package, whilst the 'mixture' model is based on 
equation \ref{eq:starmix}. 

All distributions implemented in rugarch are included as options in
\textbf{distribution.model}, while fixed and starting parameters can be passed
directly via \textbf{fixed.pars} and \textbf{start.pars} respectively, else
later on via the \textbf{setfixed<-} and \textbf{setpars<-} methods on the star
specification (note that there is also a \textbf{setbounds<-}  methods for
setting and enforcing parameter bounds).

Finally, the \textbf{fixed.probs} list allows the user to pass an xts matrix of 
fixed probabilities for each state (aligned to the index of the dataset and with
columns equal to \textbf{states}). This could for instance be the forecast
probabilities from another model (e.g. logistic regression) representing market
up and down periods, recessions etc. In this case the state equation is not used
and the model is effectively linear and extremely fast to estimate for
the conditional mean dynamics.

The returned specification object is of class \textbf{STARspec} which may be
passed to the estimation routine \textbf{starfit}. If the object has been
assigned fixed parameters for the complete parameter set, then it may instead be
passed to the \textbf{starfilter}, \code{starforecast} or the \textbf{starpath}
routines.

\subsection{Estimation}
Once the model has been specified, it may be estimated by maximum likelihood
using the \code{starfit} routine:
\begin{lstlisting}
>starfit
(spec, data, out.sample=0, solver="optim", solver.control=list(), 
fit.control=list(stationarity=0, fixed.se=0, rec.init="all"), 
cluster=NULL, n=25, ...)
\end{lstlisting}
The \textbf{data} must be an xts object with the same time indices as any data
already passed to the \code{STARspec} object and contain only numeric data
without any missing values. The \textbf{out.sample} is used to indicate how many
data points to optionally leave out in the estimation (from the end of the
dataset) for use in out-of-sample forecasting later on when the estimated object
is passed to the \code{starfilter} routine. Perhaps the most important choice
to be made is the type of \textbf{solver} to use and it's control parameters
(\textbf{solver.control}). The following solvers and 'strategies' are included:
\begin{itemize}
  \item optim. The preferred choice is the BFGS solver. The choice of solver is
  controll by the \emph{method} option in the \textbf{solver.control} list.
  \item nlminb. Have had little luck getting the same performance as the BFGS
  solver.
  \item solnp. Will most likely find a local solution.
  \item cmaes. Even though it is a global solver, it requires careful tweaking
  of the control parameters (and there are many). This is the parma package
  version of the solver.
  \item deoptim. Another global solver. May be slow and require tweaking of the
  control parameters.
  \item msoptim. A multistart version of optim with option for using the
  \textbf{cluster} option for parallel evaluation. The number of
  multi-starts is controlled by the \emph{n.restarts} option in the
  \textbf{solver.control} list.
  \item strategy. A special purpose optimization strategy for STAR problems
  using the BFGS solver. It cycles between keeping the state variables fixed and
  estimating the linear variables (conditional mean, variance and any
  distribution parameters), keeping the linear variables fixed and estimating
  the state variables, and a random re-start optimization to control for
  possibly local solutions. The argument \textbf{n} in the routine controls the
  number of times to cycle through this strategy. The \textbf{solver.control}
  list should pass control arguments for the BFGS solver. This is somewhat
  related to concentrating the sum of squares methodology in terms of the
  estimation strategy, but does not minimize the sum of squares.
\end{itemize}
The \emph{strategy} and \emph{msoptim} solver strategies should be the preferred
options when estimating STARMA models.

The resulting object of class \code{STARfit} has a number of methods including
an S4 summary (\emph{show}) and a number of S4 extractor methods such as
\code{coef}, \code{likelihood}, \code{vcov}, \code{infocriteria},
\code{modelmatrix}, \code{quantile}, \code{pit}, \code{fitted}, \code{residuals}
and \code{sigma}.
A new method \code{states} can be used for extracting the conditional state
probabilities, with an extra argument \textbf{type} with options for 'prob' 
(probabilties), 'condm' (conditional mean dynamics per state) and anything else 
will return the untransformed (raw) state dynamics. The methods are documented
in the help page of \code{STARfit-class}. A default \code{plot} method is also
available which plots the states and fitted values of the model.
The package currently does not include any substantial tests for STAR
nonlinearity or residual nonlinearity. These may be added in the future.

\subsection{Filtering}
An object of class \code{STARspec} with pre-assigned fixed parameter values
(for the complete model parameter set) may be passed to the \code{starfilter}
routine with a new or augmented dataset (to that which was used to estimate the
original parameters). In this way, new data can be filtered using existing
parameters which is equivalent to performing rolling 1-step ahead forecasts
(without re-estimation).
\begin{lstlisting}
>starfilter
(spec, data, out.sample=0, n.old=NULL, rec.init="all", ...) 
\end{lstlisting}
It is probably best to provide an augmented dataset for filtering since the
model may depend on the complete history (particularly as regards use of the
autoregressive parameter in the state dynamics), in which case the
\textbf{n.old} option should also be used to denote the size of the original
dataset.

\subsection{Forecasting}
Forecasting can be carried out either from an estimated object of class
\code{STARfit} or a specification with fixed parameters of class
\code{STARspec} (in which case the \textbf{data} argument must also be
used\footnote{Effectively, the data is filtered with the fixed parameter
specification prior to the forecast being carried out}).
\begin{lstlisting}
>starforecast
function(fitORspec, data=NULL, n.ahead=1, n.roll=0, out.sample=0, 
external.forecasts=list(xregfor=NULL, vregfor=NULL, sfor=NULL, 
probfor=NULL), method=c("an.parametric", "an.kernel", "mc.empirical", 
"mc.parametric", "mc.kernel"), mc.sims=NULL, ...) 
\end{lstlisting}
As discussed in Section \ref{sec:2}, for n-step (n>1) ahead forecasts,
there are a number of options available based on recursive quadrature
integration ('an') of the integral in Equation \ref{eq:hstepforecast} 
else monte carlo ('mc') integration. The \emph{parametric} method uses
the density from the estimated object while the \emph{kernel} method fits
a kernel density to the residuals. Finally, and only available for the monte
carlo method, the \emph{empirical} option samples from the empirical
distribution of the residuals. Clearly with a limited history it may be optimal
to use either the \emph{parametric} or \emph{kernel} methods. For the monte
carlo integration, the \textbf{mc.sims} argument denotes the number of samples
to use per period.

Some care should be taken when passing \textbf{external.forecasts} for the
conditional mean regressors (xregfor), the conditional variance regressors
(vregfor), the conditional state dynamics regressors (sfor) and the conditional
probability (probfor) in the case that the state probabilities where passed as
fixed in the specification. These xts matrices should be pre-lagged in the same
way as the input matrices where in the specification.

The resulting object is of class \code{STARforecast} with a number of extractor
functions documented in the help page of the class and similar to those
available for the \code{STARfit} class. Of particular interest in the case of
monte carlo integration is the estimated density of each point forecast which
may be extracted from the object.

\subsection{Simulation}
Simulation can be carried out either directly on a \code{STARfit} object using
the \code{starsim} routine, else on a \code{STARspec} object with fixed
parameters using the \code{starpath} method.

\begin{lstlisting}
>starsim
function(fit, n.sim=1000, n.start=0, m.sim=1, presigma=NA, prereturns=NA, 
preresiduals=NA, rseed=NA, custom.dist=list(name=NA, distfit=NA), 
xregsim=NULL, vregsim=NULL, ssim=NULL, probsim=NULL, ...)
\end{lstlisting}
The arguments follow similar convention as in related packages. In particular,
the \textbf{ssim} argument should be a list of matrices for the
simulated values of the external regressors (\textbf{s}) in the state dynamics,
while \textbf{probsim} should be a list of matrices of the simulated state
probabilities in the case that fixed probabilties were used in the original
specification.

Similarly, the \code{starpath} routine has the following arguments:
\begin{lstlisting}
>starpath
function(spec, n.sim=1000, n.start=0, m.sim=1, presigma=NA, prereturns=NA, 
preresiduals=NA, rseed=NA, custom.dist=list(name=NA, distfit=NA), 
xregsim=NULL, vregsim=NULL, ssim=NULL, probsim=NULL, ...)
\end{lstlisting}
Where the \textbf{prereturns} are now required as depending on the model, so is
\textbf{presigma} and \textbf{preresiduals}. The length of these initialization
matrices is determined by the maximum of the conditional state, mean and
variance dynamic lags.

\section{Examples}\label{sec:6}
In this section, a number of examples are considered in order to illustrate the
working of the package. A detailed testing suite is available in the src
distribution under the \textbf{inst} folder.

\subsection{The Dutch Gilder Benchmark (\cite{Franses2000})}
The package contains the \code{forex} dataset which is an xts matrix
consisting of the daily value of 8 currency pairs spanning the period
31-12-1979 to 31-12-1998 from the book of \cite{Franses2000}, and originally
sourced from the Federal Reserve Bank of New York. In Chapter 3 of their book,
they provide a model for the Dutch Gilder weekly (Wednesday) series using a STAR
model with switching based on the 4-week moving average of the lagged absolute
value of this return series.
In this
